\documentclass{article}

\usepackage[utf8]{inputenc}
\usepackage[spanish, mexico]{babel}
\usepackage[export]{adjustbox}
\usepackage{hyperref, amsmath, xcolor, mdframed, listings, tikz}
\usetikzlibrary{shapes, arrows, chains, fit, calc} 

\usepackage{showframe}


\newcommand{\crule}[2][1pt]
    {\begin{center}\rule{#2\textwidth}{#1}\end{center}}

\newcommand{\red}[1]	    {\textcolor{red}{#1}}


\newcommand\refcode[2]{ \href{#1}{\texttt{#2}} }

\newenvironment{note}
    {\begin{mdframed}[leftmargin=1cm, 
                 skipabove=1em,
                 skipbelow=1em,
                 rightline=false, 
                 topline=false,
                 bottomline=false,
                 linewidth=2pt]
        \textbf{Nota}\\}
    {\end{mdframed}}
    



\title{ \includegraphics[scale=0.6, right]{itesm_logo}
	    \\[4em] 
        Sistemas Conexionistas y Evolutivos
        \\[2em]
        $1^\text{er}$ parcial: reporte % babel shorthands don't work in macros.
        \\[1em]
        }
\author{ \vspace*{\fill} \red{ Personal Data} \qquad\small \red{Axxxxxxxx} \\[1em] }

\date{ \vspace*{\fill} \today, Puebla.}


\begin{document}
\lstset{language=Haskell, frame=single, 
    keywordstyle=\color{blue}, 
    deletekeywords={min,max},
    otherkeywords={N8}
    }

\maketitle
\newpage

\def\ImgCharRoot{run:./docs/image-characteristics}
\def\ImgCharacteristics{\ImgCharRoot/ImgCharacteristics.html}
\def\Friday{\ImgCharRoot/ImgCharacteristics-Friday.html}
\def\Extractors{\ImgCharRoot/ImgCharacteristics-Friday-Extractors.html}
\def\ExtractorBuilder{\ImgCharRoot/ImgCharacteristics-ExtractorBuilder.html}
\def\GTK{\ImgCharRoot/ImgCharacteristics-GTK.html}

\def\ExecAll{run:./docs/image-characteristics/img-chv_descriptive-stats_all/src/Main.html}

\def\Nat{run:.docs/Nat/frames.html}

\section{La base de imágenes}

Se seleccionaron 42 imagines de bosques de \underline{resoluciones diferentes}, las 36 de cuales contienen fuego forestal (o humo). Seis mas no contienen ningún signo de fuego, y tres de ellas presentan un bosque en otoño con las hojas de color. Las imagines se adjuntan en archivo \textit{images.zip}.

\begin{note}
La partición de imagines en regiones es adoptiva al tamaño de la imagen, lo que permita recibir regiones con las desviaciones de tamaño reducidas. Ademas, el vector de características se forma por los valores estadísticos de la región, los cuales no están afectados por la diferencia en el tamaño de la muestra.
\end{note}


\section{Partición de imágenes}

El proceso de partición es parte de la aplicación \verb|image-characteristics|, la cual se discuta con mas detalles el la siguiente sección.

Esta representado por un \href{http://learnyouahaskell.com/types-and-typeclasses}{\emph{clase de tipos}} \verb|RegionsExtractor| en el archivo 
\refcode{\ImgCharacteristics}{src/ImgCharacteristics.hs}.
Al momento tiene una implementación para \verb|FixedColRowRegions|, el cual describe:
\begin{enumerate}
\item el número de filas deseado (el número máximo);
\item el número de columnas deseado (el número máximo);
\item el tamaño mínimo de un \emph{región}.
\end{enumerate}

La implementación se llama \verb|fixedColRowRegions| y se encuentra en\\ \refcode{\Friday}{src/ImgCharacteristics/Friday.hs}. Utiliza la función \verb|finalSize| para encontrar el número máximo de columnas y filas, que producirán los regiones de tamaño no minor al establecido por \verb|FixedColRowRegions|. La partición de la imagen se hace con la función \href{https://hackage.haskell.org/package/friday-0.2.2.0/docs/Vision-Image-Transform.html}{\emph{crop}} de la librería \verb|friday|.

Se espera la declaración de una \href{https://downloads.haskell.org/~ghc/7.0.1/docs/html/users_guide/type-class-extensions.html}{\emph{instancia}} de \verb|RegionsExtractor| en el modulo \verb|Main|, cómo en \refcode{\ExecAll}{exec/DescriptiveStatsAll.hs} (está diseñado de esta manera para evitar posibles \href{https://wiki.haskell.org/Multiple_instances}{conflictos} de importación).

\section{Vector de características}

Un vector de características se extrae de un región de imagen por las clases \verb|CharacteristicExtractor| y \verb|CharacteristicExtractors|, los cuales definan los extractores y sus nombres. Están basados en vectores, indexados con \href{https://wiki.haskell.org/Type_arithmetic}{\emph{números naturales en nivel de tipos}}, implementados en el proyecto \refcode{\Nat}{Nat}.

Los extractores mencionados se construían utilizando \verb|ChanelExtractor| y \verb|LinkedChanelExtractor|, definidos en \\ \refcode{\ExtractorBuilder}{src/ImgCharacteristics/ExtractorBuilder.hs}.


Para el problema dado se utilizan los siguientes extractores de canales, definidos en \refcode{\Extractors}{src/ImgCharacteristics/Friday/Extractors.hs}:
\begin{itemize}
\item \verb|mean| --- la esperanza matemática;
\item \verb|meanQuadratic| --- la media cuadrática, $\sqrt{\frac{\sum\limits_{i=1}^n x_i^2}{n}}$;
\item \verb|stdev'| --- la desviación estándar, requiere valor de la media;
\item \verb|min| --- el valor mínimo ($Q_0$);
\item \verb|max| --- el valor máximo ($Q_5$);
\item \verb|quartiles| --- los cuartiles $Q_1, Q_2, Q_3$;
\end{itemize}

En total son \textbf{8} características. \\

\noindent Los extractores de canales se aplican a cada canal del región (independientemente). Un vector de características tiene dimensión \emph{número de características} $\times$ \emph{número de canales}.

\medskip

\noindent Definición de \verb|ChanelExtractor| utilizado:
\begin{lstlisting}
import ImgCharacteristics.Friday.Extractors as CE

descriptiveStats :: ( NatRules n, Floating num
                    , Ord num, GenVec n
                    , NatRules3Pack n
                    ) => ChanelExtractor n num N8
descriptiveStats =    CE.min
                  +#  CE.quartiles
                  +#  CE.max
                  +#  CE.meanQuadratic
                  +## CE.meanAndVar
\end{lstlisting}

\medskip

Para transformar \verb|ChanelExtractor| a \verb|CharacteristicsExtractor| se utilizan funciones \verb|extractorRGB|, \verb|extractorHSV| y \verb|extractorGrey|.

Se encuentran en \refcode{\Friday}{src/ImgCharacteristics/Friday.hs}.\\

En el ejecutable utilizado (\refcode{\ExecAll}{exec/DescriptiveStatsAll.hs}) se combinan los extractores \verb|RGB, HSV, Grey| para recibir vectores de dimensión \emph{56} ($8\times3 + 8\times3 + 8\times1$).

\section{Datos de aprendizaje}

Para el aprendizaje \emph{supervisado}, se requieren las etiquetas de clase para cada instancia de los datos. El funcionamiento de selección de etiquetas está definido por la clase \verb|RegionsClassesProvider| (en \refcode{\ImgCharacteristics}{src/ImgCharacteristics.hs}) y está diseñado para una acción de \href{https://www.haskell.org/tutorial/io.html}{\emph{entrada/salida}}. Provee dos funciones:
\begin{enumerate}
    \item \verb|classProvider| --- cree una instancia de \verb|RegionsClassesProvider|;
    \item \verb|regionClass| --- pregunta la clase de la imagen (región).
\end{enumerate}

Específicamente, está implementado a través de ``GIMP Toolkit'' (GTK), el cual se utiliza para el interfaz gráfico ``GNOME''. El interfaz de la ventana está proveído por un contenedor \verb|ClassesInterview|, definido en \\ \refcode{\GTK}{src/ImgCharacteristics/GTK.hs}, el cual provee:
\begin{itemize}
    \item \verb|ciWindow| --- el objeto de la ventana gráfica;
    \item \verb|ciAskClass| --- una función \verb|a -> IO class'|, dónde \verb|a| es una 
                                imagen (un región);
    \item \verb|ciDestroy| --- una función para destruir el interfaz gráfico.
\end{itemize}

\crule{1}
\medskip
Creación del \verb|ClassesInterview|:
\medskip

% see http://www.texample.net/tikz/examples/flexible-flow-chart/

\begin{tikzpicture}[%
    >=triangle 60,              % Nice arrows; your taste may be different
    node distance=20mm and 50mm, % Global setup of box spacing
    ]

\def\bind{$>>=$}

\tikzset{
  base/.style={draw, on grid, align=center, minimum height=4ex,text width=8em}, % on chain, 
  mvar/.style={base, rectangle, rounded corners}
  };

% MAIN Thread
\node[base] (main1) at (0,0) { newEmptyMVar };

\node[base, below=of main1] (forkOS) { forkOS };

\node[base, below=of forkOS, label=\emph{blocks \quad thread}] (takeMVar) { takeMVar };

\node[on grid, below=of takeMVar] (return) { \textbf{ClassesInterview} };

% MVARs
\node[mvar, right=of main1] (mVar) { MVar \\(ClassesInterview a class') };

% GUI Thread

\node[base, right=80mm of forkOS] (initGUI) { initGUI };

\node[base, below=of initGUI] (ci) {create interview};

\node[base, below=of ci] (putMVar) { putMVar };

\node[base, below=of putMVar] (mainGUI) { mainGUI };

\node[on grid, below=of mainGUI] (guiExec) { UI execution \dots };




\draw (main1.east) edge[o->] node[label=above:{\bind}]{} (mVar.west);
\draw (main1) edge[o->] node[label=right:{$>>$}]{} (forkOS);
\draw (forkOS) edge[o->] node[label=right:{$>>$}]{} (takeMVar);
\draw (takeMVar) edge[o->] (return);

\draw (forkOS) edge[->, dashed] (initGUI);
\draw (putMVar.west) edge[->, dashed, bend left] (mVar);
\draw (mVar) edge[->, dashed, bend left] (takeMVar.east);

\draw (initGUI) edge[o->] node[label=right:{$>>$}]{} (ci);
\draw (ci) edge[o->] node[label=right:{\bind}]{} (putMVar);
\draw (putMVar) edge[o->] node[label=right:{$>>$}]{} (mainGUI);
\draw (mainGUI) edge[->] (guiExec);

\end{tikzpicture}


\medskip
Interrogación del usuario \verb|extractLearnData| \\ en \refcode{ImgCharacteristics}{src/ImgCharacteristics.hs}:
\medskip



\begin{tikzpicture}[%
    >=triangle 60,              % Nice arrows; your taste may be different
    node distance=20mm and 40mm, % Global setup of box spacing
    ]

\def\bind{$>>=$}

\tikzset{
  base/.style={on grid, align=center, minimum height=4ex,text width=8em}, % on chain, 
  proc/.style={base, draw},
  mvar/.style={base, draw, rounded corners}
  };
  
% MAIN Thread
\node[base] (interview) { \emph{ClassesInterview} };

%\node[base, below= of interview] (foreachRegionIO) { foreachRegionIO };

\node[base, below=of interview] (lambdaRegion) { $\lambda \text{ region } \rightarrow$ };
\node[proc, below=10mm of lambdaRegion] (characteristics) { characteristics };
\node[proc, below=of characteristics] (regionClass) { regionClass };
\node[base, below=of regionClass] (LearnDataEntry) { \textbf{LearnDataEntry} };
\node [draw=black!50, 
       fit={(lambdaRegion) (characteristics) (regionClass) (LearnDataEntry)},
       label=above:{\emph{foreachRegionIO}} 
      ] (foreachRegionIO) {};

\node[base, below=of LearnDataEntry] (LearnDataEntries)
        { IO \\ $\left[\text{LearnDataEntry}\right]$ };

\draw[->] (interview) -- ($(foreachRegionIO.north) - (0, -6mm)$);
\draw[o->] (characteristics) -- (regionClass);
\draw[o->] (regionClass) -- (LearnDataEntry);
\draw[o->] (foreachRegionIO.south) -- (LearnDataEntries);


%AskClass

\node[base, right=of lambdaRegion] (lambdaRegion2){ $\lambda \text{ region } \rightarrow$ };
\node[proc, below=10mm of lambdaRegion2] (setImage) { setImage };
\node[proc, below=15mm of setImage] (unlockUI) { unlockUI };
\node[proc, below=15mm of unlockUI, label=\emph{blocks \quad thread}]
    (takeMVar) { takeMVar };
\node[proc, below=15mm of takeMVar] (lockUI) { lockUI };
\node[base, below=15mm of lockUI] (return) { class' };
\node[draw=black!50, 
      fit={ (lambdaRegion2) (setImage) (unlockUI) (takeMVar) (lockUI) (return)},
      label=above:{\emph{regionClass}}
     ] (regionClassGroup) {};

\draw[o->] (setImage) -- (unlockUI);
\draw[o->] (unlockUI) -- (takeMVar);
\draw[o->] (takeMVar) -- (lockUI);
\draw[o->] (lockUI)   -- (return);

% GUI Thread
\node[base, right=80mm of interview] (guiExec) { UI execution };

\node[proc, right=of setImage] (putMVar) { putMVar class' };
\node[draw=black!50, fit={(putMVar)}, label=above:{\emph{on} buttonActivated}]
    (buttonActivated) {};

\draw[->] (guiExec) -- ($(buttonActivated.north) - (0, -6mm)$);

% MVARs

\node[mvar, below=20mm of putMVar] (mVar) { MVar class' };

% Between

\draw[->, dashed] (putMVar) -- (mVar);
\draw[->, dashed] (mVar)    |- (takeMVar.east);

\draw[->, dashed] ($(regionClass.east) + (0,1mm)$) -- ($(regionClass.east) + (5mm,1mm)$) 
                                                   |- (lambdaRegion2.west);
\draw[->, dashed] (return.west) -- ($(return.west) + (-5mm,0)$)
                                |- ($(regionClass.east) - (0,1mm)$);


\node[proc, circle, right=60mm of interview, text width=2em] (user) { user };
\draw[->] (user) -- ($(buttonActivated) - (12mm,-10mm)$);

\end{tikzpicture}



\end{document}

